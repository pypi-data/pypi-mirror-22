\documentclass[12pt]{report}
\usepackage{amsmath}
\usepackage{amssymb}
\usepackage{amsfonts}
\usepackage{rotating}
\usepackage[hidelinks]{hyperref}
\usepackage{graphicx}
\usepackage{listings}
\usepackage[T1]{fontenc}
\usepackage{textcomp}
\usepackage{lmodern}
\usepackage[colorinlistoftodos,prependcaption,textsize=tiny]{todonotes}

\title{MITRE 1.0 user's manual}
\author{Eli Bogart}
\date{\today}
\begin{document}
\maketitle
\tableofcontents
%\listoffigures
%\listoftables
\lstset{
basicstyle=\small, 
stringstyle=\ttfamily,
language={},
showstringspaces=false,
frame=single,
upquote=True,
breaklines=True,
breakatwhitespace=True,
numbers=left
} 
\lstdefinelanguage{Ini}
{
    basicstyle=\ttfamily\small,
    columns=fullflexible,
    morecomment=[s][\color{blue}\bfseries]{[}{]},
    morecomment=[l]{\#},
    morecomment=[l]{;},
    commentstyle=\color{gray}\ttfamily,
    morekeywords={},
    otherkeywords={=,:},
    keywordstyle={\bfseries}
}

\chapter{Introduction}

MITRE learns interpretable predictive rules that classify hosts or
predict host outcomes based on longitudinal observations of their
microbiome.  For a full explanation of what this means, why it is
important, and how MITRE does it, refer to ``MITRE: predicting host
status from microbiota time-series data'', Elijah Bogart and Georg
Gerber (in preparation), and the associated supplementary technical
note. For a quick tutorial introduction, see the \texttt{README.md}
file, or go to \url{https://github.com/gerberlab/mitre}.  This manual
describes how to install and run MITRE, how to format your own data
for use with MITRE, what types of output are available, and the
options that may be specified in a MITRE configuration file to control
its execution.

\section{Installation}
The recommended installation procedure is to use \texttt{pip} to
download MITRE (and any dependencies not already installed) from the
Python Package Index:
\begin{quote}
  \texttt{pip install mitre}.
\end{quote}
To install from
source, run
\begin{quote}
  \texttt{git clone https://github.com/gerberlab/mitre.git}
\end{quote}
followed by
\begin{quote}
  \texttt{pip install mitre/}
\end{quote} 
(note the trailing slash.)  To
verify that installation was succcessful, run
\begin{quote}
  \texttt{mitre -{}-test}
\end{quote}
  A series of status messages should be displayed, followed by `Test
  problem completed successfully.'

You will need Python 2.7 installed to install and run MITRE. (Note
that recent versions of Python 2.7 provide the \texttt{pip} command by
default.) Only Mac and Linux systems are supported at this time.

\section{Usage}
To use MITRE, create a configuration file specifying input data, the
operations to be performed, and output types. Then run \begin{quote} \texttt{mitre
  path\_to\_configuration\_file} \end{quote} The MITRE configuration file
options, and what happens when MITRE reads a configuration file, are
described in detail in chapter \ref{configuration} below. Rather than
writing a configuration file from scratch, it may be easiest to modify
a sample configuration file to suit your needs; an example file is
provided in chapter \ref{sample} below.


\chapter{Formatting input data for MITRE}
This chapter describes, with examples, how to prepare input data files
for MITRE.

\section{Required files}
\subsection{Abundance table}\label{abundance_data}
The abundance data should be provided as a comma-separated table,
with the first row providing OTU IDs and the first column providing
sample IDs. 
\begin{lstlisting}[caption=Example abundance data table]
  "","OTU1","OTU2"
  "SampleID1",768,46
  "SampleID2",17,7545
\end{lstlisting}
(Quotes around strings are not necessary.)

\subsection{Sample metadata table}\label{sample_metadata}
The sample metadata table specifies an associated subject ID and
timepoint for each sample ID. It should be given as a comma-separated
table with no header row. No particular ordering is expected.
\begin{lstlisting}[caption=Example sample metadata table]
  SampleID1,SubjectID1,0
  SampleID2,SubjectID1,27
  SampleID3,SubjectID2,2
\end{lstlisting}

\subsection{Subject data table}\label{subject_data}
The subject data table gives information about each subject,
(including the value of whatever variable will be used as the outcome
that MITRE will try to predict, though that does not need to be
explicitly marked.) It should be given as a comma-separated table with
a header row, whose first column is the subject ID. (The first field
in the header row is ignored.) Values may be either strings or numbers
(but variables that appear to be Boolean may be converted to 0/1; see
the note for the `outcome\_positive\_value' option in section
\ref{data} below.)
\begin{lstlisting}[caption=Example subject data table]
  subjectID, delivery_mode, allergy, week_of_delivery, exposure_count
  SubjectID1, cesarean, True, 38.1, 5
  SubjectID2, vaginal, True, 39.9, 1
\end{lstlisting}

\subsection{Pplacer results}\label{jplace_file}
This should be the \texttt{.jplace} file created by placing the
representative seuqences from each OTU on a reference tree of known
16S dequences using the \texttt{pplacer} package in maximum likelihood
mode.

\section{Optional files}
\subsection{Sequence key}\label{sequence_key}
If this FASTA file of unique sequences is given, each OTU identifer in
the abundance data table that exactly matches one of the sequences
will be replaced with the identifier for that sequence in the FASTA
file. The same will be done for OTU identifiers in the placement table
described below, if it is given. This is useful when working with
tables from DADA2 that use the inferred exact sequences as
row and column headers.

\subsection{Pplacer taxonomy table}\label{pplacer_taxa_table}
A comma-separated-value taxonomy table from the \texttt{pplacer}
reference package used for placement of the OTUs. See
\url{http://fhcrc.github.io/taxtastic/refpkg.html} for a description
of the \texttt{pplacer} reference package components. The following
listing is provided as an aid to identifying the appropriate file, not
as a complete example of its necessary properties. 

\begin{lstlisting}[caption=Beginning of an example taxonomy table file]
"tax_id","parent_id","rank","tax_name","root","below_root", "superkingdom","superphylum","phylum","subphylum", "class","subclass","order","below_order", "suborder","family","below_family","subfamily", "tribe","genus","below_genus","subgenus", "species_group","species_subgroup","species"
"1","1","root","root","1","","","","","","", "","","","","","","","","","","","","",""
"131567","1","below_root","cellular organisms","1","131567","","","","","","","", "","","","","","","","","","","",""
"2157","131567","superkingdom","Archaea","1","131567", "2157","","","","","","", "","","","","","","","","","","",""
"2","131567","superkingdom","Bacteria","1","131567", "2","","","","","","","","", "","","","","","","","","",""
\end{lstlisting}

\subsection{Pplacer sequence information}\label{pplacer_seq_info}
A comma-separated-value table from the \texttt{pplacer}
reference package used for placement of the OTUs, giving information
about the sequences included in the reference package. See 
\url{http://fhcrc.github.io/taxtastic/refpkg.html} for a description
of the \texttt{pplacer} reference package components.
The following
listing is provided as an aid to identifying the appropriate file, not
as a complete example of its necessary properties. 
\begin{lstlisting}[caption=Beginning of an example sequence information file]
seqname,tax_id,species_name
S000438419,53635,Acidimicrobium ferrooxidans
S001416053,121039,Ferrimicrobium acidiphilum
S000750001,209649,Ferrithrix thermotolerans
\end{lstlisting}

\subsection{Placement table}\label{placement_table}
A comma-separated table giving predicted taxonomies for the OTUs in
the abundance table, with a header row listing the taxonomic
levels at which placements are predicted, from coarsest to finest (last two entries must be ``Genus''
and ``Species''), and with the OTU identifiers in the first column, e.g.:
\begin{lstlisting}[caption=Example placement table]
  "","Kingdom","Phylum","Class","Order","Family","Genus", "Species"
  "OTU1","Bacteria","Bacteroidetes","Bacteroidia", "Bacteroidales","Bacteroidaceae","Bacteroides","dorei"
  "OTU2","Bacteria","Verrucomicrobia","Verrucomicrobiae", "Verrucomicrobiales","Verrucomicrobiaceae", "Akkermansia",NA
\end{lstlisting}
When a placement for an OTU was not predicted at a particular level,
the entry should be NA or empty.  Note that this is the format that
results from passing the output of \texttt{dada2}'s
\texttt{addSpecies} function to \texttt{write.csv}.

\chapter{Quick guide to available MITRE output types}
MITRE offers a number of different output options. This section
provides a quick guide to how each may be specified in the configuration
file.

To generate...
\begin{itemize}
\item \textbf{An interactive visualization of the MITRE results as an
  HTML file:} Set `gui\_output' in section `postprocessing' to True.
\item \textbf{The point estimate of the best rule set, distribution of
  rule set lengths, and the Bayes factor in favor of the empty rule
  set, as a text file:} Set `quick\_summary' in section
  `postprocessing' to True.
\item \textbf{All of the above, plus the most frequent clusters of
  similar rule sets seen in the posterior distribution, as a text
  file:} Set `full\_summary' in section `postprocessing' to True.
\item \textbf{A table of Bayes factors indicating the strength of the
  evidence that particular variables or detectors are involved in the
  true rule list, as a text file:} Set `bayes\_factor\_table' in
  section `postprocessing' to True.
\item \textbf{A table describing taxonomically the OTU or group of
  bacteria corresponding to each variable in the model, as a text
  file:} Specify an appropriate `taxonomy\_source' and input data
  files in section `data', and set `aggregate\_on\_phylogeny' to True
  in section `preprocessing'.
\item \textbf{Cross-validated accuracy assessments for the MITRE point
  and ensemble estimators, as text files:} Include either a `crossvalidation' or
  `leave\_one\_out' section in the configuration file, with all
  required options given there (see sections \ref{crossvalidation} and
  \ref{leave_one_out} below.)
\item \textbf{Convergence diagnostic plots as PDF files:} Set
  `mixing\_diagnostics' in section `postprocessing' to True.
\item \textbf{The full suite of outputs necessary to compare MITRE
  performance against alternative methods:} See section
  \ref{benchmarking} below.
\end{itemize}  

Note that in almost all output text files, rules will be expressed in
terms of internal variable identifiers (which may be the names of
input OTUs, or numbers representing edges in the phylogenetic tree.)

These may be looked up in the variable annotations file (if it has
been generated) which provides a taxonomy-based label for each such
variable and, for each variable that represents a higher-level group,
a list of the OTUs included in that group.

The interactive visualization provides descriptions of the variables
visualized detector refers to; where a visualized detector represents
a higher-level group, the interface will draw the subtree of the
phylogeny corresponding to the group.

\chapter{Configuration options}\label{configuration}
\section{How a configuration file is processed}
When MITRE reads a configuration file, it attempts
to carry out each of the following operations, if the
corresponding sections of the configuration file are present:

\begin{description}
\item[preprocessing] Loading, filtering, transforming, and/or
  annotating the input data, (section \ref{preprocessing})
\item[model] Preparing a model object with specified
  parameters (section \ref{model})
\item[sampling] Drawing samples from the posterior distribution (section
  \ref{sampling})
\item[postprocessing] Generating output files, including summaries of
  the posterior distribution, mixing diagnostics, or the interactive
  visualization (section \ref{postprocessing})
\item[crossvalidation] K-fold crossvalidation: splitting the data
  across a set number of folds, creating and sampling from a model for
  each fold, assessing the accuracy of the resulting predictive rules
  for the held-out data, and writing a summary report (section \ref{crossvalidation})
\item[leave\_one\_out] As above, but performs leave-one-out
  crossvalidation instead (section \ref{leave_one_out}).
\item[comparison\_methods] Applying random forest and L1-regularized
  logistic regression models to predict the outcome, assessing their
  accuracy by cross-validation, and writing a report (section
  \ref{comparison_methods}).
\end{description}

All sections are optional. When applicable, the results of the
`preprocessing' operation will be supplied to the `model'
and `comparison\_methods' operations, the results of the `model'
step will be supplied to the `sampling' step and the
cross-validation methods, and the results of the `sampling' step
will be supplied to the `postprocessing' step.

For each operation that depends on an earlier step, however, it is
possible to oomit the earlier step and instead specify a file of
previously generated results to be loaded.

The `data' section (\ref{data}) tells MITRE where to find input data files,
and is required if the `preprocessing' section is present.

The `description' section (\ref{description}) provides information
about the problem that is used to label output files, and is
required. The `general' section (\ref{general}) is optional; it
controls options for MITRE operation that apply across all steps in
the pipeline.

The `benchmarking' section (\ref{benchmarking}) is special. It
modifies the normal sequence of operations to apply the comparison
methods to the dataset after some steps of filtering, but before the
phylogenetic aggregation process; preprocessing is then completed, and
cross-validation is performed. This is a convenience method which
facilitated the benchmarking calculations presented in the MITRE
manuscript.

\section{General}\label{general}
Options allowed in section `general' are:
\begin{description}
\item[verbose] Boolean, optional. If true, MITRE will print progress
  messages for most steps, including every MCMC iteration. Internally
  this sets the log level to \texttt{logging.INFO}. Default: false.
\end{description}

\section{Description}\label{description}
Options allowed in section `description' are:
\begin{description}
\item[tag] A short string used to generate output filenames. Required.
\end{description}

\section{Data}\label{data}
Options allowed in section `data' are:
\begin{description}
\item[load\_example] Allowed values are `bokulich', `david',
  `digiulio' or `karelia'. If one of these values is given,
  MITRE will load one of the pre-packaged example datasets, consisting
  of data from, respectively, \begin{itemize}
  \item Bokulich, N. A., et al., ``Antibiotics, birth mode, and diet
    shape microbiome maturation during early life'', \textit{Science
      Translational Medicine} 8(343): 343ra82 (2016)
  \item David, L. A., et al., ``Diet rapidly and reproducibly alters
    the human gut microbiome'', \textit{Nature} 505(7484): 559 -- 563
    (2014)
  \item DiGiulio, D. B., et al., ``Temporal and spatial variation of
    the human microbiota during pregnancy'', \textit{PNAS} 112(35):
    11060 -- 11065 (2015)
  \item Vatanen, T., et al., ``Variation in Microbiome LPS
    Immunogenicity Contributes to Autoimmunity in Humans'',
    \textit{Cell} 165(4): 842 -- 853 (2016),
\end{itemize}
reprocessed as described in the MITRE supplementary note.  If this
option is given, no other options in this section are required. Note
that the `tag' option, above, is overwritten in this case.
\item[abundance\_data] Filename or path to a table of OTU abundance
  data, formatted as described in section
  \ref{abundance_data}. Required, unless `load\_example' is set.
\item[sample\_metadata]
  Filename or path to a table giving a subject and timepoint for each
  sample in the abundance table, formatted as described in section
  \ref{sample_metadata}. Required, unless `load\_example' is set.
\item[subject\_data]
  Filename or path to a table of data about each subject,
  formatted as described in section
  \ref{subject_data}. Required, unless `load\_example' is set.
\item[jplace\_file] Filename or path to pplacer results in
  \texttt{.jplace} format, as described in section
  \ref{abundance_data}. Required if `aggregate\_on\_phylogeny' is set
  to True in section `preprocessing', or `taxonomoy\_source' is
  `pplacer` or `hybrid', unless `load\_example' is set.
\item[sequence\_key]
  Filename or path to a FASTA file to be used to rename
  the OTUs listed in the abudance table, in the case where
  the name of each OTU is simply a sequence, as in some DADA2 output;
  see \ref{sequence_key}. Optional.
\item[outcome\_variable] String, specifying which of the columns in
  the subject data table encodes the outcome that MITRE should try to
  predict. Subjects for whom this data is not available will be
  dropped from the calculation.  Required, unless `load\_example' is
  set.
\item[outcome\_positive\_value] The value of the outcome variable that
  corresponds to a positive outcome. All other (non-missing) values
  will be considered a negative outcome. Due to a minor limitation of
  MITRE's data parsing system, \textit{if the outcome variable is
    given as `true'/`false', `True'/`False', etc., in the subject data
    table, the positive value must be given as `1' or `0' in the
    configuration file.}  Other strings or numerical values should
  work as expected. Required, unless `load\_example' is set.
  
\item[taxonomy\_source] String specifying how taxonomic annotations
  for each variable (i.e., subtree of the overall phylogeny) should be
  generated. Valid options are `table', `pplacer', and `hybrid' (see
  the supplement to the MITRE manuscript for a full explanation of
  these options; in short, `table' labels OTUs from a table and higher
  clades based on the OTUs they contain, `pplacer' labels all
  variables based on the pplacer results, and `hybrid' labels
  variables based on pplacer results except where a species-level
  placement is given in a table.) If this option is not present, no
  such annotation will be done. If given, note that the table of
  annotations will be written to
  \texttt{[tag]\_variable\_annotations.txt} at the conclusion of the
  preprocessing step, unless phylogenetic aggregation is not
  performed.
    
\item[pplacer\_taxa\_table] Filename or path to a table from the
  pplacer reference package, formatted as described in section
  \ref{pplacer_taxa_table}. Required if `taxonomy\_source' is
  `pplacer' or `hybrid'; otherwise ignored.
\item[pplacer\_seq\_info] Filename or path to a table from the pplacer
  reference package, formatted as described in section
  \ref{pplacer_seq_info}. Required if `taxonomy\_source' is `pplacer'
  or `hybrid'; otherwise ignored.

\item[additional\_subject\_covariates] String or comma-separated list
  of strings specifying additional variables (columns in the subject
  data table) to be included in the regression calculation. See
  ``Incorporating additional covariates'', section 1.6 of the
  supplementary note to the MITRE manuscript for full details. These
  variables should be categorical. For quantitative covariates, see
  below. Subjects where data for any of these variables
  are missing will be dropped from the calculation. Optional.
\item[additional\_covariate\_default\_states] String or
  comma-separated list of strings specifying, in the appropriate
  order, the values of the additional covariates to be treated as the
  default state. For each such covariate, the model will incorporate a
  change in the odds of the outcome associated with every other value
  of the covariate, but not this value. Required if
  `additional\_subject\_covariates' is specified.
\item[additional\_subject\_continuous\_covariates] String or
  comma-separated list of strings specifying additional variables
  (columns in the subject data table) to be included in the regression
  calculation as quantitative covariates. The name is somewhat
  misleading, as integer-valued variables are fine here. See
  ``Incorporating additional covariates'', section 1.6 of the
  supplementary note to the MITRE manuscript for full
  details. Subjects where data for any of these variables are missing
  will be dropped from the calculation. Optional.
\end{description}

\section{Preprocessing}\label{preprocessing}
See section 2.1 of the MITRE supplementary note for a detailed
discussion of the data preprocessing and filtering procedure
controlled by the `preprocessing' section of the configuration
file. Note that the ordering of the options in the configuration file
does not matter.  Steps will be carried out in the order the
corresponding options are presented below, skipping any for which the
corresponding options are not given in the configuration file. Except
where specified, any of these options may be omitted.

\begin{description}
\item[min\_overall\_abundance] Numeric value ($N_\text{counts,OTU}$ in
  the notation of section 2.1). If this is specified, all OTUs with
  lower total abundance data than this value, summed across all
  samples, are dropped. (Abundance data is assumed to be measured in
  16S read counts at this stage, but this is not enforced.)
\item[min\_sample\_reads] Numeric value ($N_\text{counts,sample}$ in
  the notation of section 2.1). If this is specified, all samples
  where the total abundance data across all (remaining) OTUs sum to
  less than this value are dropped. (Abundance data is assumed to be
  measured in 16S read counts at this stage, but this is not
  enforced.)
\item[trim\_start] Numeric value ($t_i$ in the notation of section
  2.1). The time to consider as the beginning of the study. (By
  default, this will be the timepoint of the earliest (remaining)
  sample.) Samples before this time will be dropped. If this is given,
  `trim\_stop' must be given also.
\item[trim\_stop] Numeric value ($t_f$ in the notation of section
  2.1). The time to consider as the end of the study. (By
  default, this will be the timepoint of the latest (remaining)
  sample.) Samples after this time will be dropped.
\item[density\_filter\_n\_samples] Numeric value ($N_s$ in the
  notation of section 2.1). If this is given,
  `density\_filter\_n\_intervals' and
  `density\_filter\_n\_consecutive' must be given also. The duration
  of the study will be divided into the specified number of time
  intervals, and subjects from whom at least the specified number of
  samples are available in \textit{every} time window formed from the
  specified number of consecutive intervals are retained; samples from
  all other subjects are dropped.
\item[density\_filter\_n\_intervals] Integer ($N_\text{w,filter}$ in
  the notation of section 2.1). See above.
\item[density\_filter\_n\_consecutive] Integer ($N_c$ in the notation
  of section 2.1). See above.
\item[take\_relative\_abundance] Boolean. If True, abundance data will
  be converted to relative abundances by normalizing so that the sum
  of the data for each sample is 1.0.
\item[aggregate\_on\_phylogeny] Boolean. If True, introduce new
  variables corresponding to subtrees of the overall phylogeny of the
  OTUs, inferred from \texttt{pplacer} results given in the
  `jplace\_file' option above, as discussed in section 2.2 of the
  supplement. The data for each new variable will be the sum of the data
  for the OTUs contained in the subtree.
\item[log\_transform] Boolean. If True, take the natural log of all
  abundance data. Values less than $10^{-10}$ will be replaced by
  $10^{-6}$ before taking the log.
\item[temporal\_abundance\_threshold] Numeric value ($a$ in the
  notation of section 2.1).  If this option is given,
  `temporal\_abundance\_consecutive\_samples' and
  `temporal\_abundance\_n\_subjects' must be also. Variables which
  exceed the abundance threshold in at least the specified number of
  consecutive samples in the data from each of at least the specified
  number of subjects will be kept; others will be dropped. If the data
  has been log-transformed, this should be specified on a log scale.
\item[temporal\_abundance\_consecutive\_samples] Numeric value ($N_a$
  in the notation of section 2.1). See above.
\item[temporal\_abundance\_n\_subjects] Numeric value ($N_i$ in the
  notation of section 2.1). See above.
\item[discard\_surplus\_internal\_nodes] Boolean. If true,
  variables corresponding to subtrees rooted at internal nodes which
  have only one (remaining) child node after the filtration steps
  have been applied will be dropped from the analysis.
\item[pickle\_dataset] Boolean. If true, the MITRE datastructure
  corresponding to the processed and filtered dataset will be
  serialized to \texttt{[tag]\_dataset\_object.pickle} at the
  conclusion of the preprocessing step.
\end{description}

\section{Model}\label{model}
\subsection{Required values}
Required or conditionally required options section `model' are:
\begin{description}
\item[load\_data\_from\_pickle] String giving the location of a
  \texttt{dataset\_object.pickle} file from a previous preprocessing
  step. Must be given if no `preprocessing' section in this file. If
  given, a model for that dataset will be built; otherwise, a model
  will be built for the result of the `preprocessing' step specified
  in this file.
\item[n\_intervals] Integer, required. ($N_w$ in the notation of
  chapter 1 of the MITRE supplementary note.) The study duration will
  be divided into this many intervals and rules will apply to time
  windows formed one or more consecutive such intervals.
\item[t\_min] Numeric value, required. ($W_\text{min}$ in the notation of
  chapter 1 of the MITRE supplementary note.) Rules may apply only
  to time windows this long or longer.
\item[t\_max] Numeric value, required. ($W_\text{max}$ in the notation of
  chapter 1 of the MITRE supplementary note.) Rules may apply only
  to time windows this long or shorter.
\end{description}

\subsection{Saving the model object}
An optional setting allows saving the model for reuse:
\begin{description}
\item[pickle\_model] Boolean. If true, the MITRE datastructure
  representing the model will be serialized to
  \texttt{[tag]\_model\_object.pickle} at the conclusion of the model
  setup step.
\end{description}

\subsection{Advanced settings}
The following advanced options, all numeric, are not required. See the
indicated sections of the MITRE
supplementary note for explanations of their significance.
\begin{description}
\item[prior\_coefficient\_variance] $\sigma_b^2$ in the notation of
  section 1.3. (default 100.0)
\item[hyperparameter\_alpha\_m] $\alpha_m$ in the notation of section 1.4 (default 0.5)
\item[hyperparameter\_beta\_m] $\beta_m$ in the notation of section 1.4 (default 2.0)
\item[hyperparameter\_alpha\_primitives] $\alpha_n$ in the notation of section 1.4 (default 2.0)
\item[hyperparameter\_beta\_primitives] $\beta_n$ in the notation of section 1.4 (default 4.0)
\item[hyperparameter\_a\_empty] $a_\Theta$ in the notation of section
  1.4.1 (default 0.5)
\item[hyperparameter\_b\_empty] $b_\Theta$ in the notation of section
  1.4.1 (default 0.5)
\item[max\_thresholds] $N_\theta$ in the notation of section 1.2.4
  (default, no maximum)
\item[max\_rules] $m_\text{max}$ in the notation of section 1.4 (default 10)
\item[max\_primitives] $n_\text{max}$ in the notation of section 1.4 (default 10)
\item[window\_concentration\_typical]
  $c_{w,\text{typical}}=1/\lambda_w$ in the notation of section 1.4.3
  (default 5.0)
\item[window\_concentration\_update\_ratio] Controls the
  standard deviation of the proposal distribution for the
  update $\delta_w$ to $c_W$ relative to $c_{w,\text{typical}}$
  (default 0.2); see section 1.5.4.
\item[delta\_l\_scale\_mean] Variance of the normal
  prior distribution of $\mu_L$ in the notation of section 1.4.3, relative
  to $\Delta_L^2$ (default 50.0). 
\item[delta\_l\_scale\_sigma] Upper bound on the range of allowed
  values for $\sigma_L$ in the notation of section 1.4.3, relative to
  $\Delta_L$ (default 25.0)
\item[lambda\_l\_offset] Amount by which $\Lambda_L$ in the notation
  of section 1.4.3 differs from the median of the logarithms of the
  subtree weights $L_i$ (default 0.0).
\end{description}

\section{Sampling}\label{sampling}
Options allowed in section `sampling' are:
\begin{description}
\item[load\_model\_from\_pickle] String giving the location of a
  \texttt{model\_object.pickle} file from a previous model setup
  step. Must be given if no `model' section in this file. If given,
  that model will be sampled from; otherwise, the model specified by
  the `model' section of this file will be used.
\item[total\_samples] Total number of samples to draw. Either this or
  `sampling\_time' must be given.
\item[sampling\_time] Number of seconds to run the sampling process. If present,
  overrides `total\_samples'.
\item[pickle\_sampler] Boolean. If true, the MITRE datastructure
  representing the sampler will be serialized to
  \texttt{[tag]\_sampler\_object.pickle} at the conclusion of
  sampling.
\end{description}

\section{Postprocessing}\label{postprocessing}
The following settings, all optional, may be specified in section `postprocessing':
\begin{description}
  \item[load\_sampler\_from\_pickle] String giving the location of a
    \texttt{sampler\_object.pickle} file from a previous sampling
    step. Must be given if no `sampling' section in this file. If
    given, generate output from the samples drawn by that sampler;
    otherwise, from the samples drawn as specified in the `sampling'
    section of this file.
  \item[burnin\_fraction] Number between 0 and 1 giving the fraction of
    samples to discard as burn-in before analysis. Defaults to 0.05. 
  \item[quick\_summary] Boolean. If True, write the point summary,
    the distribution of rule set lengths, and the Bayes factor in
    favor of the empty rule set 
    \texttt{[tag]\_quick\_summary.txt}.
  \item[full\_summary] Boolean. If True, write the point summary,
    the distribution of rule set lengths, the Bayes factor in
    favor of the empty rule set, and the highest-posterior-probability
    rule set clusters (see section 3.3 of the MITRE supplement) to 
    \texttt{[tag]\_full\_summary.txt}.
  \item[bayes\_factor\_table] Boolean. If True, write the 10 variables
    and 10 detectors with the highest Bayes factors supporting their
    inclusion in the true model, and those factors, to
    \texttt{[tag]\_bayes\_factor\_table.txt}. See section 3.4 of the
    MITRE supplement.
  \item[gui\_output] Boolean. If true, prepare an interactive
    visualization of the results as described in section 3.5 of the
    MITRE supplement and save it to
    \texttt{[tag]\_visualization.html}. Note, this output cannot be
    generated if `aggregate\_on\_phylogeny' is not set.
  \item[bayes\_factor\_samples] Integer. Draw this many samples
    to estimate the prior distribution over the space of detectors in the
    process of estimating Bayes factors, if the other options in this
    section require the estimation of Bayes factors (beyond the
    Bayes factor for the empty rule set.) Default 10000.
  \item[mixing\_diagnostics] Boolean. If true, plot the sampled values
    of the likelihood, prior, rule set length (total number of
    detectors), an arbitrary subset of the auxiliary variables
    $\omega$, and parameters $f_w$, $c_w$, $\mu_L$, and $\sigma_L$
    (defined in the MITRE supplementary note) versus MCMC
    iteration. Results will be saved in
    \texttt{[tag]\_likelihood.pdf}, \texttt{[tag]\_prior.pdf}, etc.
    
\end{description}

\section{K-fold crossvalidation}\label{crossvalidation}
Required or conditionally required options in section `crossvalidation' are:
\begin{description}
\item[load\_model\_from\_pickle] String giving the location of a
  \texttt{model\_object.pickle} file from a previous model setup
  step. Must be given if no `model' section in this file. If given,
  that model will be crossvalidated; otherwise, the model specified by
  the `model' section of this file will be used.
\item[n\_folds] Integer; how many folds of cross-validation should be
  done.
\item[parallel\_workers] Integer; how many parallel worker processes
  should be launched. 
\item[burnin\_fraction] Number between 0 and 1 giving the fraction of
  samples to discard as burn-in before calculating accuracy metrics
  for each fold.
\item[total\_samples] Total number of samples to draw for each
  fold. Either this or `sampling\_time' must be given.
\item[sampling\_time] Number of seconds to run the sampling process
  for each fold. If present, overrides `total\_samples'.
\end{description}

Non-mandatory options are:
\begin{description}
\item[write\_reports\_every\_fold] Boolean. If True, the quick summary
  described in section \ref{postprocessing} will be written after
  sampling concludes for every fold (\texttt{[tag]\_0\_qr.txt},
  \texttt{[tag]\_1\_qr.txt}, etc.) Note that accuracy metrics in these
  reports are calculated based on the \textit{training} set for the fold,
  not the held-out test set.
\item[write\_full\_summaries\_every\_fold] Boolean. If True, the full
  summary described in section \ref{postprocessing} will be written
  after sampling concludes for every fold
  (\texttt{[tag]\_0\_full\_summary.txt},
  \texttt{[tag]\_1\_full\_summary.txt}, etc.) Note that accuracy
  metrics in these reports are calculated based on the
  \textit{training} set for the fold, not the held-out test set.
\item[pickle\_results] Boolean. If true, the true outcomes and
  predicted probabilities of a positive outcome for each subject in
  each test set under the point and ensemble classifiers will be
  written (as serialized Python objects) to
  \texttt{[tag]\_cv\_point\_results.pickle} and
  \texttt{[tag]\_cv\_ensemble\_results.pickle} for further analysis.
\end{description}
At the conclusion of crossvalidation, a table of accuracy metrics for
the point and ensemble classifiers learned in each fold, applied to
the held-out test data for each fold, will be written to
\texttt{[tag]\_mitre\_cv\_report.txt}.

\section{Leave-one-out crossvalidation}\label{leave_one_out}
The options for section `leave\_one\_out' are the same as those for
section `crossvalidation', above, except that `n\_folds' is not
required (and is ignored if given.)

At the conclusion of crossvalidation, a table of accuracy metrics for
the predictions made on the left-out subjects for each fold will be
written to \texttt{[tag]\_mitre\_leave\_one\_out\_report.txt}.


\section{Comparison}\label{comparison_methods}
This section supports benchmarking the MITRE models against
alternative methods.  See the MITRE manuscript for the details of the
comparison methods used and an explanation of these parameters.  Options
in section `comparison\_methods' are:
\begin{description}
\item[load\_data\_from\_pickle] String giving the location of a
  \texttt{dataset\_object.pickle} file from a previous preprocessing
  step. Must be given if no `preprocessing' section in this file. If
  given, a model for that dataset will be built; otherwise, a model
  will be built for the result of the `preprocessing' step specified
  in this file.
\item[cv\_type] String, optional. If 'leave\_one\_out', assess the performance
  of the methods by leave-one-out crossvalidation; otherwise,
  use k-fold crossvalidation.
\item[n\_folds] Integer, required unless `cv\_type' is `leave\_one\_out',
  specifying how many folds to use for k-fold crossvalidation.
\item[n\_intervals] Integer, required. $N_\text{w,comparison}$ in the
  notation of the MITRE manuscript.
\item[n\_consecutive] Integer, required. $N_\text{c,comparison}$ in
  the notation of the MITRE manuscript.
\end{description}

\section{Benchmarking}\label{benchmarking}
If this section is present:
\begin{enumerate}
\item Data is loaded following the options in the `preprocessing' section
    normally, through the step of conversion to relative abundance. 

\item The first comparison-methods crossvalidation is run immediately,
  following the options specified in the `comparison' section of the
  configuration file as usual. (If log-transformation is specified in
  the `preprocessing' section, this is applied to the data passed to the
  comparison methods.) The resulting report will be labeled
  \texttt{[tag]\_benchmark\_step3\_comparison.txt}.

\item Temporal filtering options are applied to a copy of the dataset
  and a second comparison-methods crossvalidation is run. (Again, if
  log-transformation is specified in the `preprocessing' section, this
  is applied to the data passed to the comparison methods.)  The
  resulting report will be labeled
  \texttt{[tag]\_benchmark\_step4\_comparison.txt}.

\item All later actions in the `preprocessing' section are applied to
  the dataset normally, such as aggregation on the phylogenetic tree
  and log transformation. A third comparison-methods crossvalidation
  is run. The resulting report will be labeled
  \texttt{[tag]\_benchmark\_step5\_comparison.txt}.
    
\item The dataset is used for MITRE crossvalidation, following the
  options specified in the `model' and `crossvalidation' or
  `leave\_one\_out' section of the configuration file as usual.
\end{enumerate}

The comparison method F1 scores reported in the manuscript are those
from \texttt{...step3\_comparison.txt} or \texttt{...step4\_comparison.txt},
whichever are better, to provide the most favorable conditions for the
comparison methods.

One option is allowed in this section (but its use is
not recommended, as it is likely to lead to confusion.)
\begin{description}
\item[load\_data\_from\_pickle] String giving the location of a
  \texttt{dataset\_object.pickle}, \textit{preprocessed up to
    conversion to relative abundance, but not farther.} If given, this
  will be loaded and the first part of the preprocessing stage will be
  skipped; options governing later steps of the preprocessing
  operation must still be specified in this file.
\end{description}

\chapter{Example configuration file}\label{sample}
This file is also available as \texttt{example.cfg} in the MITRE
source distribution. It specifies that MITRE should load data from the
dietary perturbation study of David et al (2014), reprocessed as
described in the MITRE supplementary note, learn rules that will
predict whether a subject ate the exclusively plant-based diet or not,
and produce text and interactive summaries of the results.

This file cannot be used exactly as is: you will need to edit the
paths to the data files, and change the outcome and filtering
parameters to whatever is appropriate for your own
study. (Alternatively, if you do want to rerun this calculation, you
can simply replace the options in the `data' section with
\texttt{load\_example = david} to use the copies of the relevant files
included with the MITRE distribution.)

\begin{lstlisting}[language=Ini,caption=Template MITRE configuration file,numbers=none]
# Note: comments placed on their own lines
# may start with ';' or '#', but 
# inline comments should start only with ';'.

[description]
tag = david_diet

[data]
abundance_data = /path/to/your_data/abundance.csv
sequence_key = /path/to/your_data/sequence_key.fa
sample_metadata = /path/to/your_data/sample_metadata.csv
subject_data = /path/to/your_data/subject_data.csv
jplace_file = /path/to/your_data/placements.jplace
outcome_variable = diet
outcome_positive_value = Plant
taxonomy_source = hybrid
pplacer_taxa_table = /path/to/your_data/taxaTable.csv 
pplacer_seq_info = /path/to/your_data/seq_info.csv
placement_table = /path/to/your_data/mothur_placements.csv

[preprocessing]
# Note that the order of these options in the configuration file is
# not important. Those filters and transformations selected will be
# applied in the following order: 
# overall abundance filter, sample depth filter, window trimming,
# temporal sampling density filter, relative abundance
# transformation, phylogenetic aggregation, log transform,
# temporal abundance filter, surplus internal node filtering.

# Drop RSVs with less than a certain number of reads across
# all samples
min_overall_abundance = 10
# Drop samples with less than a certain total number of reads
min_sample_reads = 5000
# No need to trim the time window here, -5 to 10 is okay
# trim_start = 20
# trim_stop = 30
# Discard subjects with insufficiently dense temporal sampling:
# divide the window up into n_intervals equal pieces, 
# then require at least n_samples within every n_consecutive
# consecutive such pieces
# We don't actually need to do this here, the sampling density
# is good for all 20 subjects.
# density_filter_n_samples = 1
# density_filter_n_intervals = 12
# density_filter_n_consecutive = 2
# Select which transformations should be applied to the data. 
take_relative_abundance = True
aggregate_on_phylogeny = True
log_transform = False
# Temporal abundance filtration: keep only those taxa which 
# exceed a threshold abundance in multiple consecutive observations
# at least once in a minimum number of subjects. Note that the 
# threshold should be on a log scale if the log transform has been 
# performed.
temporal_abundance_threshold = 0.001
temporal_abundance_consecutive_samples = 2
temporal_abundance_n_subjects = 4
# Discard taxa representing nodes in the phylogenetic tree not
# required to maintain the topological relationships among
# the other nodes of the tree still included in the model.
discard_surplus_internal_nodes = True
# Save this dataset as a pickle file?
pickle_dataset = True

[model]
# Divide the experiment into this many equal segments and use them
# as the atomic time windows
n_intervals = 10
# Allow rules to apply only to time windows longer than t_min
t_min = 1.0
# Allow rules to apply only to time windows shorter than t_max
t_max = 7.0
pickle_model = True

[sampling] 
total_samples = 50000

[postprocessing]
quick_summary = True
full_summary = True
gui_output = True
burnin_fraction = 0.1
\end{lstlisting}


\end{document}
